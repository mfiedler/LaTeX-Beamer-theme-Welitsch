% Copyright 2013 by Michael Fiedler <michael.fiedler87@gmx.de>
%
% This file may be distributed and/or modified
%
% 1. under the LaTeX Project Public License and/or
% 2. under the GNU Public License.
\documentclass[12pt]{beamer}
\usepackage[utf8]{inputenc}
\usepackage{graphicx}
\usepackage{hyperref}
\usepackage{url}
\usetheme{Welitsch}

\title{\LaTeX{} Beamer theme ``Welitsch''\footnote{\url{http://github.com/mfiedler/LaTeX-Beamer-theme-Welitsch}}}
\subtitle{Example slides}
\author{Michael Fiedler\footnote{\url{mailto:michael.fiedler87@gmx.de}}}
\date{\today}

% Show the table of contents before each section
\AtBeginSection[]{
	\begin{frame}
		\frametitle{Outline}
		\tableofcontents[currentsection]
	\end{frame}
}

\begin{document}

\maketitle


\begin{frame}
	\frametitle{Outline}
	\tableofcontents
\end{frame}


\section{Some section}
\begin{frame}
\frametitle{Some content}
\framesubtitle{A subtitle}

Here are the three different block types:

\begin{block}{Block}
Some text
\end{block}

\begin{exampleblock}{Example block}
Some text
\end{exampleblock}

\begin{alertblock}{Alert block}
Some text. \alert{ALERT!}
\end{alertblock}
\end{frame}

\begin{frame}
\frametitle{Itemize}

\begin{itemize}
\item some item
\begin{itemize}
\item a subitem
\begin{itemize}
\item a subsubitem
\item another subsubitem
\end{itemize}
\item another subitem
\end{itemize}
\item another item
\item last item
\end{itemize}
\end{frame}


\begin{frame}
\frametitle{Enumerations and cross references}

Nested enumerations:

\begin{enumerate}
\item first item
\begin{enumerate}
\item foo
\item bar \label{bar}
\begin{enumerate}
\item first subitem
\item second subitem
\end{enumerate}
\item more text
\begin{enumerate}
\item foobar
\item labeled item \label{foobar}
\end{enumerate}
\end{enumerate}
\item second first-level item \label{foo}
\begin{enumerate}
\item some
\item text
\end{enumerate}
\end{enumerate}

You can also cross reference the enumerate items, e.\,g. \ref{foo}, \ref{bar} or \ref{foobar}

\end{frame}


\begin{frame}
\frametitle{Welitsch}

\begin{figure}
\centering
\includegraphics[width=0.4\textwidth]{welitsch_coat_of_arms}
\caption{
Coat of arms of Welitsch\cite{commons:welitsch}
}
\end{figure}
\end{frame}


\begin{frame}
\frametitle{Questions?}
\begin{center}
\Huge \textbf{Questions?}
\end{center}
\begin{center}
\url{http://github.com/mfiedler/LaTeX-Beamer-theme-Welitsch}
\end{center}
\end{frame}


\section{References}
\begin{frame}{References}
\bibliographystyle{plain}
\bibliography{literature}
\end{frame}

\end{document}
